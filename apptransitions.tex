\subsection{Application Transitions}
\label{subsec-apptransitions}

In any computing system, studying its workload is a key to improve the overall
performance. By analyzing how applications behave, we can observe common usage
patterns that arise and optimize relevant components to exploit those patterns.
This is even more crucial for smartphones as they are still
resource-constrained, and yet users run resource-heavy applications such as 3D
games.

Joint use of multiple applications is potentially one interesting usage pattern.
This might arise in scenarios such as checking social networking applications in
series and playing different games in one seating. If there is a group of
applications with a high correlation in usage, then a phone OS might treat them
together and apply the same scheduling policy, i.e., the applications can be
loaded into the memory together to reduce the latency of context switch.
A previous study has looked at a similar usage pattern by analyzing a network
traffic trace~\cite{xu:imc:2011}. Our study described below is a more direct
analysis based on recorded application usage events, hence more fine-grained.

\subsubsection{Jointly-Used Applications}
In order to understand which applications are used together, we calculate a {\it
transition} probability from an application $a$ to another application $b$
within a session. A session is defined as the time between a screen unlock event
and the subsequent screen lock event. To calculate a transition probability, we
count the number of start events for $a$, i.e., {\it start(a)}, and the number
of start events for $b$ that occur after an $a$'s start event within the same
session, i.e., {\it start(a $\rightarrow$ b)}. Our transition probability is
$\frac{start(a \rightarrow b)}{start(a)}$.

\subsubsection{Future Experiments}
