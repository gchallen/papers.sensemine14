Although smartphones have emerged as the most significant large-scale mobile
platform in computing history, the scale of smartphone experimentation has
lagged behind. Keeping pace requires new facilities that enable
experimentation at scale to ensure that research discoveries translate to the
growing network of smartphone devices.

% 02 Doc 2012 : GWA : Testbed size is frontend/models.py:UserDevice.working()

This paper introduces \PhoneLab{}, a 191~device smartphone testbed deployed
at SUNY Buffalo. \PhoneLab{} provides access to an incentivized group of
participants ready to engage in Android experimentation. The testbed will
open for public experimentation in October, 2013, and grow to over
700~\XXXnote{GWA: We may want to revisit this number.} smartphones by 2014.

% 02 Dec 2012 : GWA : run on ... phones is
% ./figures/statistics/lib.py % --experiment_count

% 02 Dec 2012 : GWA : for ... days is
% ./figures/statistics/lib.py % --experiment_length_days
% For initial abstract submission this number is estimated.

To demonstrate the power of \PhoneLab{} we present three selected results
from a usage characterization experiment run on 115 phones for 21
days~\XXXnote{GWA: Need to update these numbers.}. We use each result to
motivate a future \PhoneLab{} experiment, demonstrating the power of
\PhoneLab{} to enable mobile systems research.
