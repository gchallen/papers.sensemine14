\section{Related Work}
\label{sec:related}

Testbeds in other domains have chosen their design points
to meet domain-specific needs. For example, PlanetLab~\cite{peterson:ccr:2003,
planetlab} operates more than 1,000 machines world-wide in order to enable
large-scale, realistic Internet research. Emulab~\cite{white:osdi:2002, emulab}
provides emulated network environments to enable controlled, repeatable network
experiments. MoteLab~\cite{werner-allen:ipsn:2005} targets realistic sensor
network experiments by deploying a sensor network testbed in a building at
Harvard. ORBIT~\cite{raychaudhuri:tridentcom:2005} takes a two-tier approach
allowing emulated experiments as well as real deployments, targeting
reproducibility and realism at the same time.
OpenCirrus~\cite{avetisyan:computer:2010, opencirrus} and VICCI~\cite{vicci} are
geographically distributed clusters, designed to support cloud computing
research. \PhoneLab{}, on the other hand, aims to provide a large-scale,
realistic platform for smartphone experimentation, resulting in a number of
different design points as described in this paper.

%{\bf Measurement Tools:} Researchers have developed special-purpose
%measurement tools for smartphone usage measurement. These tools are designed to
%measure certain metrics such as power consumption~\cite{zhang:codes:2010,
%pathak:eurosys:2012} or 3G performance~\cite{huang:mobisys:2010}. Though we have
%developed a measurement tool for our analysis as well, our focus for this paper
%is not the tool design, but the power of \PhoneLab{}.
%
%{\bf Smartphone Measurement Studies:}
%Though the primary purpose of our analysis is to demonstrate the power of
%\PhoneLab{}, our findings complement what previous studies have reported. Falaki
%et al.~\cite{falaki:mobisys:2010} are among the first ones to study smartphone
%usage. Their central finding is that in many of the metrics they studied, there
%was significant diversity without a clear pattern; the metrics include the mean
%interaction length, the mean number of applications used, the mean amount of
%traffic, etc. Xu et al.~\cite{xu:imc:2011} use a network-level trace to analyze
%smartphone application usage. The key findings are that smartphone users use
%many regional applications such as local news apps; certain applications are
%installed together; and mobility patterns affect the types of applications
%used. Trestian et al.~\cite{trestian:imc:2009} study users of a 3G network and
%report a similar finding: people's movement and locations correlate with
%applications they use. They also report that there is a correlation between
%content upload and location in another study~\cite{trestian:ton:2012}. Shye et
%al.~\cite{shye:micro:2009} studies how power consumption is distributed over
%different hardware components. Kim et al.~\cite{kim:fast:2012} focus on
%correlation between storage and application performance and find multiple
%factors that affect application performance. Many studies have also looked at
%Wifi and 3G network characteristics using
%smartphones~\cite{keralapura:mobicom:2010, maier:pam:2010, gember:pam:2011}.
