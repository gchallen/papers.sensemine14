\section{Related Work}
\label{sec:related}

Most similar to \PhoneLab{} are the NetSense~\cite{netsense-hotplanet}
project at Notre Dame and the LiveLabs~\cite{livelabs-url} testbed at
Singapore Management University. NetSense has many similarities to
\PhoneLab{}: it distributed instrumented smartphones to several hundred
incoming freshman undergraduate students. In contrast to \PhoneLab{},
however, NetSense was built to support a single study---on how use of digital
technologies impacts friendship formation---and was never designed or
operated as a public testbed. LiveLabs is a city-scale research testbed
designed to allow companies to run large-scale consumer trials and experiment
with novel services. It aims to recruit thousands of participants,
potentially providing scale exceeding that of \PhoneLab{}, but is also not
public.

Other computer science testbeds meet domain-specific needs.
PlanetLab~\cite{peterson:ccr:2003} operates more than 1,000 machines
world-wide to facilitate large-scale, realistic Internet research.
Emulab~\cite{white:osdi:2002} provides emulated network environments to
enable controlled, repeatable network experiments.
MoteLab~\cite{werner-allen:ipsn:2005} provided access to 200 sensor network
nodes deployed in a multi-story office building.
ORBIT~\cite{raychaudhuri:tridentcom:2005} takes a two-tier approach allowing
emulated experiments as well as real deployments, targeting reproducibility
and realism at the same time. OpenCirrus~\cite{avetisyan:computer:2010} is a
geographically distributed cluster designed to support cloud computing
research.
