\section{Related Work}
\label{sec:related}

We have chosen the design points of \PhoneLab{}---scale, realism, locality, and
platform experimentation---in order to meet the needs of smartphone research
that are currently lacking. To the best of our knowledge, \PhoneLab{} is the
only smartphone testbed that plans to provide all four characteristics.

LiveLabs~\cite{livelabs} is the most similar testbed that we are aware of. Its
plan is to provide scale, realism, and locality, but lacks the ability to
perform platform experimentation. LiveLabs will eventually have three sites---a
mall, a campus, and a theme park---that are instrumented with custom wireless
infrastructure; and it will collect location and activity information from the
participants. Since it is a work in progress, the exact features and operational
details are not yet documented well.

As a case study, we demonstrate the power of \PhoneLab{} by collecting and
analyzing comprehensive usage data from a subset of our participants
(Section~\ref{evaluation}). Though smartphone usage measurement studies exist,
our goal is to showcase that \PhoneLab{} enables {\it any} researcher to
examine {\it real, up-to-date} smartphone usage from a large pool of
participants {\it according to their needs when they need it}; with the lack of
such ability, researchers have resorted to arduous methods such as obtaining and
analyzing service providers' network-level data sets~\cite{xu:imc:2011,
trestian:imc:2009, trestian:ton:2012}, incentivizing a large number of
participants themselves~\cite{falaki:mobisys:2010}, recruiting volunteers
through a long period of publicizing the project~\cite{shye:micro:2009}, etc.
Although these processes make it possible to obtain traces, they are
prohibitedly expensive to repeat, hence the traces quickly become obsolete.
Through our case study, we demonstrate that \PhoneLab{} greatly simplifies this
process.

\XXXnote{stevko: need to revise according to our findings.}
Though the primary purpose of our analysis is to demonstrate the power of
\PhoneLab{}, our findings complement what previous studies have reported. Falaki
et al.~\cite{falaki:mobisys:2010} are among the first ones to study smartphone
usage. Their central finding is that in many of the metrics they studied, there
was significant diversity without a clear pattern; the metrics include the mean
interaction length, the mean number of applications used, the mean amount of
traffic, etc. Xu et al.~\cite{xu:imc:2011} use a network-level trace to analyze
smartphone application usage. The key findings are that smartphone users use
many regional applications such as local news apps; certain applications are
installed together; and mobility patterns affect the types of applications
used. Trestian et al.~\cite{trestian:imc:2009} study users of a 3G network and
report a similar finding: people's movement and locations correlate with
applications they use. They also report that there is a correlation between
content upload and location in another study~\cite{trestian:ton:2012}. Shye et
al.~\cite{shye:micro:2009} studies how power consumption is distributed over
different hardware components. Kim et al.~\cite{kim:fast:2012} focus on
correlation between storage and application performance and find multiple
factors that affect application performance. Many studies have also looked at
WiFi and 3G network characteristics using
smartphones~\cite{keralapura:mobicom:2010, maier:pam:2010, gember:pam:2011}.

Researchers have also developed special-purpose measurement tools. These tools
are designed to measure certain metrics such as power
consumption~\cite{zhang:codes:2010, pathak:eurosys:2012} or 3G
performance~\cite{huang:mobisys:2010}. Though we have developed a measurement
tool for our analysis as well, our focus for this paper is not the tool design,
but the power of \PhoneLab{}.

Testbeds in other domains also have chosen their design points to meet
domain-specific needs. For example, PlanetLab~\cite{peterson:ccr:2003,
planetlab} operates more than 1,000 machines world-wide in order to enable
large-scale, realistic Internet research. Emulab~\cite{white:osdi:2002, emulab}
provides emulated network environments to enable controlled, repeatable
network experiments. MoteLab~\cite{werner-allen:ipsn:2005} targets realistic
sensor network experiments by deploying a sensor network testbed in a building
at Harvard. ORBIT~\cite{raychaudhuri:tridentcom:2005} takes a two-tier approach
allowing emulated experiments as well as real deployments, targeting
reproducibility and realism at the same time.
