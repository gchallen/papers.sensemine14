\section{Related Work}
\label{sec:related}

Most similar to \PhoneLab{} are the NetSense~\cite{netsense-hotplanet}
project at Notre Dame and the LiveLabs~\cite{FIXME-livelabs} testbed at
Singapore Management University. NetSense has many similarities to
\PhoneLab{}: it distributed instrumented smartphones to several hundred
incoming freshman undergraduate students. In contrast to \PhoneLab{},
however, NetSense was built to support a single study---on how use of digital
technologies impacts friendship formation---and never designed or operated as
a public testbed. LiveLabs is a city-scale research testbed designed to allow
companies to run large-scale consumer trials and experiment with novel
services. It aims to recruit thousands of participants, potentially providing
scale exceeding that of \PhoneLab{}, but is also not public.

Testbeds in other domains have chosen their design points to meet
domain-specific needs. PlanetLab~\cite{peterson:ccr:2003} operates more than
1,000 machines world-wide in order to enable large-scale, realistic Internet
research. Emulab~\cite{white:osdi:2002} provides emulated network
environments to enable controlled, repeatable network experiments.
MoteLab~\cite{werner-allen:ipsn:2005} targets realistic sensor network
experiments by deploying a sensor network testbed in a building at Harvard.
ORBIT~\cite{raychaudhuri:tridentcom:2005} takes a two-tier approach allowing
emulated experiments as well as real deployments, targeting reproducibility
and realism at the same time. OpenCirrus~\cite{avetisyan:computer:2010} and
VICCI~\cite{vicci} are geographically distributed clusters, designed to
support cloud computing research.
