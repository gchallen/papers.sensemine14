\section{Introduction}
\label{sec-introduction}

Smartphones have quickly become the most popular computing platform. Google
announced in September, 2013, that it had activated over 1~billion Android
devices in only six years, with 500~million in the last year
alone~\cite{google-total-activations}. The International Data Corporation
(IDC) projects that 224~M smartphone units will ship worldwide in 2013 Q4, a
40\% increase over 2012 Q4~\cite{idc-smartphone-growth}. Taken as a whole,
the growing network of smartphone devices represents the largest and most
pervasive distributed system ever built.

Unfortunately, the scale of smartphone experimentation is not keeping pace. A
quick survey of MobiSys'12, MobiSys'13, and SenSys'13 papers reveals that
frequently smartphone evaluations on real devices use small numbers of
phones---for example, 2, 3, or 7~\cite{nowar-mobisys12, wang:mobisys:2013,
hao:sensys:2013}. Many other experiments use simulations driven by small,
old, or synthesized data sets~\cite{falcon-mobisys12, ace-mobisys12,
humanmobility-mobisys12}. In either case, large-scale results from real users
would be more compelling. While user recruitment, human subjects compliance,
and reliable data collection make large-scale smartphone experimentation
challenging, harnessing the growth of smartphones requires evaluating ideas
at scale.

In this paper, we present \PhoneLab{}, a large public smartphone testbed
hosted by the University at Buffalo that enable research at scales currently
impractical. \PhoneLab{} provides the features necessary for smartphone
research:

\begin{itemize}[nosep,leftmargin=0.15in,labelwidth=\itemindent]
\vspace*{-0.1in}

\item \textbf{Scale:} \PhoneLab{} currently has 288 participants incentivized
to join experiments. Unlike applications distributed on the Play Store,
\PhoneLab{} experiments do not need to benefit or even interact with users,
facilitating research unlikely to be popular on application marketplaces.

\item \textbf{Power:} By utilizing the Android open-source smartphone
platform, \PhoneLab{} allows interactive experiments, non-interactive data
collection, and changes to the Linux kernel; Android middleware and
libraries; and Dalvik virtual machine. Passive experimentation is difficult
to perform without platform support, and platform experiments are impossible
to distribute on application marketplaces.

\item \textbf{Realism:} Participants use the phones as their primary device.

\item \textbf{Relevance:} \PhoneLab{} allows researchers to stop relying on
out-of-date datasets. Instead, new data can be collected in the most
appropriate way for the experiment.

\vspace*{-0.1in}
\end{itemize}

Section~\ref{sec-testbed} describes \PhoneLab{}, including its design,
implementation, and data collection mechanism. Next, we demonstrate that
\PhoneLab{} is powerful and usable in Section~\ref{sec-experiments} by giving
three example data analysis results. For this purpose, we have conducted a
usage measurement experiment run by 115 \PhoneLab{} participants for six
months. Rather than attempting a comprehensive analysis of the dataset, we
use it to highlight the power of \PhoneLab{} and breadth of research it
supports by presenting three results:

\begin{itemize}[nosep,leftmargin=0.15in,labelwidth=\itemindent]
\vspace*{-0.1in}

\item overall energy breakdown (Section~\ref{subsec-energybreakdown}),

\item opportunistic charging (Section~\ref{subsec-opportunistic}),

\item 3G to Wifi transitions (Section~\ref{subsec-networktransitions}).

\vspace*{-0.1in}
\end{itemize}

These examples are intended to encourage others in the mobile systems
community to use \PhoneLab{}.
