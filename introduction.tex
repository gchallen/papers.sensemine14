\section{Introduction}
\label{sec-introduction}

Smartphones have become the most popular computing platform. Google reports
1.3~M Android device activations per day in September,
2013~\cite{google-Sep2012-activations}, while IDC projects that 224~M smartphone
units will ship worldwide in 2013 Q4, a 40\% increase over 2012
Q4~\cite{idc-smartphone-growth}. Taken as a whole, the growing network of
smartphone devices represents the largest, most pervasive distributed system in
history.

Meanwhile, the scale of smartphone experimentation is not keeping pace. A small
survey of MobiSys'12, MobiSys'13, and SenSys'13 papers reveals that often times
when smartphone evaluations use real devices, they use small numbers of
phones---for example, 2, 3, or 7~\cite{nowar-mobisys12, wang:mobisys:2013,
hao:sensys:2013}. Many other experiments use simulations driven by small, old,
or synthesized data sets~\cite{falcon-mobisys12, ace-mobisys12,
humanmobility-mobisys12}. In either case, large-scale results from real users
would be more compelling. While multiple factors---including recruitment, human
subjects compliance, and data collection---make large-scale smartphone
experimentation challenging, harnessing the growth of smartphones requires
evaluating new ideas at scale.

In this paper, we present \PhoneLab{}, a large programmable smartphone testbed
that enables smartphone research at scales currently impractical. \PhoneLab{}
provides access to a large and stable set of participants incentivized to
participate in smartphone experimentation; \PhoneLab{} is continually growing
with 191~participants in 2012, and 288~participants in 2013. With this scale,
\PhoneLab{} increases the density and interaction rate between participants,
facilitating the evaluation of phone-to-phone protocols and crowd-sourcing
algorithms.

\PhoneLab{} provides the features necessary for smartphone research---power,
scale, realism, locality, and relevance:

\begin{itemize}
\item {\bf Scale:} \PhoneLab{} currently has 288 participants, already recruited
to participate in experiments.
\item {\bf Power:} \PhoneLab{} allows application-level experiments as well
as platform-level, i.e., Android kernel, middleware, libraries, and Dalvik
virtual machine.
\item {\bf Realism:} Participants use the phones as their primary device.
\item {\bf Locality:} Most participants live in Buffalo near SUNY campuses,
enabling research requiring device-to-device interaction.
\item {\bf Relevance:} \PhoneLab{} allows researchers to stop relying on
out-of-date datasets. Instead, new data can be collected in the most
appropriate way for the experiment.
\vspace{0.05in}
\end{itemize}

By utilizing the Android open-source smartphone platform, \PhoneLab{} enables
research above and below the platform interface. Researchers can distribute
new interactive applications or non-interactive data loggers, but can also
change core Android platform components, allowing \PhoneLab{} to host systems
experiments impossible to distribute through the Play Store.

We first describe \PhoneLab{} in Section~\ref{sec-testbed}, its current design,
implementation, and the data collection mechanism. We then demonstrate in
Section~\ref{sec-experiments} that \PhoneLab{} is powerful and usable by giving
three example data analysis results. For this purpose, we have conducted a usage
measurement experiment run by 115 \PhoneLab{} participants for 21
days~\XXXnote{GWA: FIXME.}. Rather than attempting a comprehensive analysis of
the dataset, we use it to highlight the power of \PhoneLab{} and breadth of
research it supports. We present three results on:

\begin{itemize}
\item overall energy breakdown (Section~\ref{subsec-energybreakdown}),
\item opportunistic charging (Section~\ref{subsec-opportunistic}),
\item 3G to Wifi transitions (Section~\ref{subsec-networktransitions}).
\vspace{0.05in}
\end{itemize}

We hope that these examples will encourage \PhoneLab{} use by the mobile systems
community.
